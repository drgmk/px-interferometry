\documentclass[11pt]{article}

\usepackage{geometry}                % See geometry.pdf to learn the layout options. There are lots.
%\geometry{a4paper}                   % ... or a4paper or a5paper or ... 
%\geometry{landscape}                % Activate for for rotated page geometry
%\usepackage[parfill]{parskip}    % Activate to begin paragraphs with an empty line rather than an indent
\usepackage{graphicx}
\usepackage{amssymb}
%\usepackage{epstopdf}
%\DeclareGraphicsRule{.tif}{png}{.png}{`convert #1 `dirname #1`/`basename #1 .tif`.png}

\title{Optical interferometry}
\author{Grant Kennedy}
%\date{}                                           % Activate to display a given date or no date

\begin{document}
\maketitle

\section{Introduction}

You are probably familiar with Young's double slit experiment, where light from a single monochromatic point source (commonly a laser) passes through two slits and forms ``fringes'' on a screen or detector. The spacing of these fringes is $\lambda/b$, where $\lambda$ is the wavelength of the light, and $b$ the distance between the slits. The bright fringes are where the path lengths from the two slits are equal, so the light constructively adds, and the dark fringes are where the paths differ by half of the wavelength of the light, and so cancel.

There is a somewhat obvious question to ask of this experiment; what happens if the source is not a point? One reason this is obvious is that sources are never really points, and have some finite size. For example, we can for most purposes consider stars to be point sources because their angular size is very small compared to the resolution of typical telescopes, but given sufficient angular resolution this assumption is not true.

A finite source size fundamentally changes the double slit experiment. Consider a second point source of equal brightness that is incoherent with the first, and that is located at an angle $\lambda/(2b)$ away. The two fringe patterns are $180^\circ$ out of phase so will cancel out\footnote{This is incoherent combination of light, so we average over the random relative phases from the two sources}, leaving a uniformly illuminated detector.

While either source would individually produce a set of fringes, together their fringes interfere to produce none, and this difference is the result of the spatial separation of the two sources. That is, the double slit experiment therefore provides a means to infer spatial information about sources by measuring fringe patters. Here we will be concerned with the fringe amplitudes, which are also known as ``visibilities'', but the phases are also important.


\end{document}  